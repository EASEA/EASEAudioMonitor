\documentclass{article}
\usepackage[francais]{babel}
\usepackage[utf8]{inputenc}
\begin{document}

  \title{Stage musicalisation pour EASEA: \\
    \large Rapport 2}
  \author{PALLAMIDESSI Joseph}
  \maketitle

  \newpage
  \section{Récapitulatif} % (fold)
  \label{sec:Récapitulatif}
    \paragraph{} % (fold)
    \label{par:}
    Comme énoncé lors du rapport précédent, je me suis focalisé durant ces deux
    semaines sur la partie purement "mécanique" du système. Comme nous l'avons conclu
    suite à la réunion du lundi 17 février, nous procéderons de manière relativement
    empirique à la réalisation sonore selon les contraintes énoncées (pas fatiguant
    pour l'oreille, musicalement intéressant, etc ... ).\\
    Cette approche repose donc sur le fait d'avoir un système ou workbench déjà en 
    place.
    % paragraph  (end)
  % section Récapitulatif (end)
  
  \section{Avancement du système} % (fold)
  \label{sec:Avancement du système}
    \paragraph{} % (fold)
    \label{par:}
      Le système de monitoring d'EASEA est déjà en place et fonctionnel. Il a été
      conçu dans l'idée d'être flexible et générique, pour justement pouvoir
      beaucoup expérimenter. Je peux donc dès maintenant me lancé dans le
      coeur du sujet, l'algorithme de composition.\\
      Pour l'instant, j'ai mis en place une gamme de Shepard et un rythme de Risset 
      dont la vitesse est lié à la moyenne des meilleurs individus, mais le résultat
      final est douteux et très (impossible) à analyser (les changements de vitesse
      sont relativement faible).
    % paragraph  (end)
  % section Avancement du système (end)

  \section{Considération technique} % (fold)
  \label{sec:considération technique}
    \paragraph{} % (fold)
    \label{par:}
      Pour le coté technique, j'ai rajouté un module de communication TCP/IP aux
      programmes compilés par Easea, qui notifie un serveur central lors de chaque
      génération et lors d'une réception/envoi d'individu. Ce serveur central "compose"
      alors selon les données qu'il reçoit et envoie des directives spécifiques au
      programme/serveur de synthèse supercollider qui joue le rôle "d'interprète".
      Le serveur central intègre un petit module statistique, pour notamment
      normaliser les données des clients EASEA.\\
      Je fournirai bientôt les diagrammes de classe et autres UML (dans une semaine
      environ).
    % paragraph  (end)
  % section considération technique (end)

  \section{Prochaine étape} % (fold)
  \label{sec:Prochaines étapes}
    \paragraph{} % (fold)
    \label{par:}
      Pour dans deux semaines, j'aimerai avoir une sonification (musicalisation déjà
      ?) basique mais utilisable.\\
      Je reste ouvert à toute les suggestions sur le plan musical comme technique,
      et je serais ravi de répondre a vos questions.
    % paragraph  (end)
  % section Prochaines étapes (end)

\end{document}
