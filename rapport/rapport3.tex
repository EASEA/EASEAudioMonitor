\documentclass{article}
\usepackage{array}
\usepackage[francais]{babel}
\usepackage[utf8]{inputenc}
\begin{document}

  \title{Stage musicalisation pour EASEA: \\
    \large rapport3}
  \author{PALLAMIDESSI Joseph}
  \maketitle
  \newpage

  \section{Récapitulatif} % (fold)
  \label{sec:Récapitulatif}
    \paragraph{} % (fold)
    \label{par:}
    
      Ces deux dernières semaines, j'ai créer des outils qui seront particulièrement
      utile à l'expérimentation. En particulier un simulateur de cloud EASEA, qui
      utilise les logs de runs. Cela permettra d'accélérer grandement les phase de tests.\\
      J'ai aussi commencer des systèmes de musicalisation (et de sonification), que je
      vais détailler plus loin.\\
      La principal avancée, sur le plan théorique, et le fait que j'ai pu trouver un
      premier système de notation des nœuds du cloud.
    % paragraph  (end)
  % section Récapitulatif (end)

  \section{Notation} % (fold)
  \label{sec:Notation}
    \paragraph{} % (fold)
    \label{par:}
      Sans plus attendre, voici le système de notation.\\
      A chaque génération d'un nœud, nous récupérons la note de son meilleur
      individu, de son pire individu, la note moyenne de la population et
      l'écart type de la population. Certains paramètres sont plus important que
      d'autres, notamment l'écart type et le meilleur individu. \\
      Nous classons
      donc les variations de paramètres entre deux générations selon ces critères
      d'importance et nous obtenons le tableaux suivant: 
    % paragraph  (end)
    \paragraph{} % (fold)
    \label{par:}
      \begin{tabular}{|c|c|c|c|}
        \hline
        stdev & average & best & worst \\
        \hline
        \nearrow & \searrow & \searrow & \searrow \\
        \hline
        - & \searrow & \searrow & - \\
        \hline
        \nearrow & \searrow & - & - \\
        \hline
        - & - & - & - \\
        \hline
      \end{tabular}
    % paragraph  (end)  
    \paragraph{} % (fold)
    \label{par:}
      La logique derrière ces choix est assez naïve: quand toute les variables
      évoluent dans le "bon" (ou plutôt attendu) sens de variation, il s'agit du
      meilleurs cas possible et au contraire si entre deux générations rien ne bouge
      alors le nœud à "planter" ou a été particulièrement inefficace. \\
      Il y a une
      ambiguïté quand au déclenchement de la notation, qui peut se faire par un 
      temporisateur, ou de manière événementielle, chacune des ces méthodes ayant des
      défauts et des avantages différents (avec temporisateur: note régulière dans
      le temps , mais pas
      moyen de distinguer entre un plantage et un nœud qui pédale dans la semoule).
    % paragraph paragraph name (end)
  % section Notation (end)

  \section{Systèmes en cours de développement} % (fold)
  \label{sec:Systèmes en cours de développement}
    \paragraph{} % (fold)
    \label{par:}
      Il y a trois systèmes "musicaux" en cours de développement, qui auront des
      résultats plus ou moins cohérent. 
    % paragraph  (end)
      \subsection{Par automate cellulaire} % (fold)
      \label{sub:Par automate cellulaire}
        \paragraph{} % (fold)
        \label{par:}
        
          Le système paramètre à son tour un système complexe (Jeu de la Vie). L'idée
          peut sembler aberrante, mais en utilisant certaines structures (notamment
          oscillateur et gun) et en les associant (fixant) à l'état des nœuds, on
          pourra peut-être obtenir une sonification intéressante (pas de musique ici
          !).
        % paragraph  (end)
      % subsubsection Par automate cellulaire (end)
      \subsection{Par lecture de partition} % (fold)
      \label{sub:Par lecture de partition}
       \paragraphe} % (fold)
       \label{par:paragraph name}
        
         Ici les nœuds deviennent interprètes, une "partition" (MIDI) est fournie au
         départ et selon leurs notes, ils jouent plus ou moins bien (mal...) : \\un
         demi-ton au dessus, un temps avant/après etc... \\Dans la pratique ce sera
         peut-être le système le plus agréable/performant, mais on perd le côté
         compositionnel du sujet. De plus de nouvelles contraintes apparaissent:\\
         Quel type de partitions faut il fournir, sont elles facile à trouver/créer ?
         etc...
       % paragraph paragraph name (end)
      % subsubsection Par lecture de partition (end)

      \subsection{Par chaîne de Markov} % (fold)
      \label{sub:Par chaîne de Markov}
        \paragraph{} % (fold)
        \label{par:}
        
          L'idée est de se baser sur une matrice de transition entre des éléments
          musicaux. Les probabilités de passer d'un état à l'autre changerons (comment
          ?) selon l'état du système. Il s'agit d'un vision extrêmement naïve de la
          musique, qu'il faudra forcément agrémenter d'un support.
        % paragraph  (end)  
      % subsubsection Par chaîne de Markov (end)
  % section Systèmes en cours de développement (end)

  \section{Jusqu'à la prochaine étape} % (fold)
  \label{sec:Jusqu'à la prochaine étapes}
    Je suis actuellement train d'implémenter le système basé sur le jeu de la vie,
    je devrais avoir fini d'ici dans 2-3 jours, et si le résultat n'est pas
    complètement dysfonctionnel je vous enverrai des samples. Si une de ces idées
    vous semble particulièrement intéressante, ou au contraire irréalisable,
    n'hésiter pas à prendre contact, je reste à votre entière disposition.
  % section Jusqu'à la prochaine étapes (end)
\end{document}
\end{article}
